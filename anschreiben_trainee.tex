\documentclass[11pt,a4paper]{letter}
 
\usepackage[utf8]{inputenc} % for unicode typing
\usepackage[ngerman]{babel} % Neue Deutsch-Rechtschreibung
\usepackage[T1]{fontenc} % fonts encoding
\usepackage[]{newtxtext} % arial font

\usepackage{graphicx} % insert image and picture
\usepackage[left=2.5cm,
            right=2cm,
            top=2cm,
            bottom=2cm]{geometry} % define pagelayout

% using microtype package, so to text could be better format
\usepackage[activate={true,nocompatibility},
            final,
            tracking=true,
            kerning=true,
            spacing=true,
            factor=1100,
            stretch=10,
            shrink=10]{microtype}

\usepackage{bookmark} % create a custom bookmark

\setlength{\parskip}{1em} % set the space between two paragraph
\setlength{\parindent}{0pt} % set the indent space of paragraph
\setcounter{secnumdepth}{2} % set the depth of the document, section

\usepackage{xargs} % use argument in new command
\usepackage[table,dvipsnames]{xcolor} % add color support for text

% TODO notes
\usepackage[colorinlistoftodos,prependcaption,textsize=footnotesize]{todonotes} % write TODO
\newcommandx{\unsure}[2][1=]{\todo[linecolor=red, backgroundcolor=red!25, bordercolor=red, #1]{#2}}
\newcommandx{\change}[2][1=]{\todo[linecolor=blue, backgroundcolor=blue!25, bordercolor=blue, #1]{#2}}
\newcommandx{\info}[2][1=]{\todo[linecolor=OliveGreen, backgroundcolor=OliveGreen!25, bordercolor=OliveGreen, #1]{#2}}
\newcommandx{\improvement}[2][1=]{\todo[linecolor=Plum, backgroundcolor=Plum!25, bordercolor=Plum, #1]{#2}}

% macros, define in documents
\newcommand{\FirmaName}{Aumann Espelkamp GmbH}
\newcommand{\FirmaAdresseLineOne}{In der Tütenbeke 37}
\newcommand{\FirmaAdresseLineTwo}{32339 Espelkamp}

\newcommand{\JobName}{Konstruktion und Entwicklung}
\newcommand{\JobBereich}{}
\newcommand{\JobRefNummer}{WO 71030}

\newcommand{\AnsprechpartnerVoll}{Frau Mareike Köllmann}
\newcommand{\Ansprechpartner}{Maier}
\newcommand{\Ansprechpartnerin}{Köllmann}

\begin{document}
% Manual import bookmark
\bookmark[page=1, level=-2]{Anschreiben}
% pagestyle empty means no page number
\pagestyle{empty}

% Sender
\begin{flushleft}
    \textbf{Ngoc Minh Dao}\\
    Bömelburgstraße 18B\\
    30165 Hannover\\
    E-Mail: \href{mailto:ngocminhdao88@gmail.com}{ngocminhdao88@gmail.com}\\
    Tel.: (+49) 176 74561589
\end{flushleft}

% Empfänger
\begin{flushleft}
    \textbf{\FirmaName}\\
    \AnsprechpartnerVoll\\
    %Personalabteilung\\
    \FirmaAdresseLineOne\\
    \FirmaAdresseLineTwo\\
\end{flushleft}

% Today
\begin{flushright}
    Hannover, \today
\end{flushright}

\textbf{Bewerbung als Trainee im Bereich Konstruktion}\\

%Sehr geehrte Damen und Herren,
%Sehr geehrter Herr \Ansprechpartner,
Sehr geehrte Frau \Ansprechpartnerin,

bei der diesjährigen Career Dates-Firmenkontaktmesse der Leibniz Universität Hannover verschaffte ich mir einen Überblick über Ihr Unternehmen und bin auf Ihr Trainee-Programm aufmerksam geworden.
Da ich meine berufliche Zukunft im Bereich des Sondermaschinenbaus sehe, möchte ich gern Ihr Konstruktionsteam am Standort Espelkamp verstärken.

Im Juni habe ich mein Studium des Maschinenbaus an der Leibniz Universität Hannover abgeschlossen.
Schon während meines Studiums konnte ich zahlreiche praktische Erfahrungen im technischen Bereich sammeln.
Als HiWi beim Institut für Maschinenkonstruktion und Tribologie unterstützte ich die wissenschaftlichen Mitarbeiter bei Konstruktionsaufgaben, der Durchführung von Versuchen für verschiedene Wälzlager-Forschungsprojekte sowie dem Aufbau und der Programmierung (LabVIEW) von Prüfständen.
Als Werkstudent bei der Micreon GmbH habe ich die Mitarbeiter bei der Reparatur und der Konstruktion von elektronischen und mechanischen Laborinstrumenten unterstützt.
Meinem Interesse an der Technik folgend, habe ich im Rahmen meiner Projektarbeit ein Prüfkonzept zur Untersuchung der Dichtwirkung von Radial-Wellendichtringen unter Fliehkrafteinfluss ausgearbeitet und konstruktiv realisiert.
In meiner Abschlussarbeit habe ich ein modulares Messsystem zur optischen und elektrischen Schmierfilmdickenmessung im EHD-Kontakt entwickelt und in Betrieb genommen.

Weitere industrielle Erfahrungen konnte ich in einem sechsmonatigen Praktikum bei der Schaeffler Technologies AG \& Co sammeln.
Hier unterstützte ich das Getriebe-Anwendungstechnik-Team bei der Berechnung von Fahrzeuggetrieben und der Schadensanalyse von Lagern.

Neben dem Studium erweiterte ich meine technischen Kenntnisse durch Online-Kurse (Creo, Solidworks, ANSYS, Python etc.).
Außeruniversitär engagierte ich mich als Mitglied in einer akademischen Gruppe für Kraftfahrwesen und unterstützte Kommilitonen sowohl bei der Wartung als auch der Reparatur von Fahrzeugen.

In die neuen Aufgaben bei Ihnen kann ich verschiedene Stärken einbringen.
Meine Aufgaben habe ich stets zuverlässig, verantwortungsbewusst, präzise und gewissenhaft erledigt.
Insbesondere Flexibilität, Motivation und Teamfähigkeit zeichnen mich aus.
In meinen Praktika und Hiwi-Jobs konnte ich bereits meine hohe Lernbereitschaft und Kreativität unter Beweis stellen.

Über Ihre Einladung zu einem persönlichen Gespräch würde ich mich sehr freuen.

\vspace{1em}

Mit freundlichen Grüßen

%\includegraphics[scale=0.2]{./signaturen/NgocMinhDao-Signatur.pdf}\\
Ngoc Minh Dao

\end{document}
