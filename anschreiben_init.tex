\documentclass[11pt,a4paper]{letter}
 
\usepackage[utf8]{inputenc} % for unicode typing
\usepackage[ngerman]{babel} % Neue Deutsch-Rechtschreibung
\usepackage[T1]{fontenc} % fonts encoding
\usepackage[]{newtxtext} % arial font

\usepackage{graphicx} % insert image and picture
\usepackage[left=2.5cm,
            right=2cm,
            top=2.5cm,
            bottom=2cm]{geometry} % define pagelayout

% using microtype package, so to text could be better format
\usepackage[activate={true,nocompatibility},
            final,
            tracking=true,
            kerning=true,
            spacing=true,
            factor=1100,
            stretch=10,
            shrink=10]{microtype}

\usepackage{bookmark} % create a custom bookmark

\setlength{\parskip}{1em} % set the space between two paragraph
\setlength{\parindent}{0pt} % set the indent space of paragraph
\setcounter{secnumdepth}{2} % set the depth of the document, section

\usepackage{xargs} % use argument in new command
\usepackage[table,dvipsnames]{xcolor} % add color support for text

% TODO notes
\usepackage[colorinlistoftodos,prependcaption,textsize=footnotesize]{todonotes} % write TODO
\newcommandx{\unsure}[2][1=]{\todo[linecolor=red, backgroundcolor=red!25, bordercolor=red, #1]{#2}}
\newcommandx{\change}[2][1=]{\todo[linecolor=blue, backgroundcolor=blue!25, bordercolor=blue, #1]{#2}}
\newcommandx{\info}[2][1=]{\todo[linecolor=OliveGreen, backgroundcolor=OliveGreen!25, bordercolor=OliveGreen, #1]{#2}}
\newcommandx{\improvement}[2][1=]{\todo[linecolor=Plum, backgroundcolor=Plum!25, bordercolor=Plum, #1]{#2}}

% macros, define in documents
\newcommand{\FirmaName}{WABCO GmbH}
\newcommand{\FirmaAdresseLineOne}{Am Lindener Hafen 21}
\newcommand{\FirmaAdresseLineTwo}{30453 Hannover}

\newcommand{\JobName}{Konstruktion und Entwicklung}
\newcommand{\JobBereich}{}
\newcommand{\JobRefNummer}{WO 71030}

\newcommand{\AnsprechpartnerVoll}{Frau Ellena Domes}
\newcommand{\Ansprechpartner}{Maier}
\newcommand{\Ansprechpartnerin}{Domes}

\begin{document}
% Manual import bookmark
\bookmark[page=1, level=-2]{Anschreiben}
% pagestyle empty means no page number
\pagestyle{empty}

% Sender
\begin{flushleft}
    \textbf{Ngoc Minh Dao}\\
    Bömelburgstraße 18B\\
    30165 Hannover\\
    E-Mail: \href{mailto:ngocminhdao88@gmail.com}{ngocminhdao88@gmail.com}\\
    Tel.: (+49) 176 74561589
\end{flushleft}

% Empfänger
\begin{flushleft}
    \textbf{\FirmaName}\\
    %\AnsprechpartnerVoll\\
    Personalabteilung\\
    \FirmaAdresseLineOne\\
    \FirmaAdresseLineTwo\\
\end{flushleft}

% Today
\begin{flushright}
    Hannover, \today
\end{flushright}

%\textbf{Initiativbewerbung als Ingenieur}
\textbf{Initiativbewerbung um eine Stelle als Ingenieur im Bereich: Konstruktion und Entwicklung}\\
\vspace{1em}

Sehr geehrte Damen und Herren,
%Sehr geehrter Herr \Ansprechpartner,
%Sehr geehrte Frau \Ansprechpartnerin,

am Tag des Career Dates im Jahr 2018 verschaffte ich mir einen Überblick über Ihr Unternehmen und war von Ihren Portfolios sehr beeindruckt.
Daher war ich sehr erfreut, als ich über einen Bekannter (Herr Phillip Bautz) erfuhr, dass Sie ständig auf der Suche nach neuen qualifizierten und motivierten Ingenieuren sind.

Schon während meines Studiums konnte ich praktische Erfahrungen im technischen Bereich sammeln.
Als HiWi beim Institut für Maschinenkonstruktion und Tribologie unterstützte ich den wissenschaftlichen Mitarbeitern bei der Konstruktionsaufgabe, dem Aufbau, der Programmierung (LabVIEW) von Prüfständen sowie der Durchführung von Versuchen für verschiedene Wälzlager-Forschungsprojekte.

Das Studium des Maschinenbaus habe ich meine Schwerpunkte auf Konstruktion/Entwicklung und Fahrzeugsysteme gelegt.
Meinem Interesse an der Technik folgend, habe ich im Rahmen meiner Masterarbeit ein modulares Messsystem zur optischen und elektrischen Schmierfilmdickenmessung im EHD-Komtakt entwickelt.
Neben dem Studium erweiterte ich meine technischen Kenntnisse dadurch, dass ich an Online-Kursen (Projektmanagement, ProE/Creo, Python etc.) teilnahm.
Außeruniversitär war ich ein Mitglied in einer akademischen Gruppe für Kraftfahrwesen, wo ich in meiner Freizeit anderen Kommilitonen bei der Wartung bzw. der Reparatur von Fahrzeugen unterstützte.

Weitere praktische Erfahrungen in Industrie konnte ich in einem sechsmonatigen Praktikum bei der Schaeffler Technologies AG \& Co sammeln.
Hier unterstützte ich dem Getriebe-Anwendungstechnik-Team bei der Berechnung von Fahrzeuggetrieben, der Schadensanalyse von Lagern, sowie dem Empfang und dem Büromanagement.
Hierbei konnte ich meine Teamfähigkeit unter Beweis stellen.

Ich kann Deutsch und Englisch sprechen und mache gern außer der Arbeit die gemeinsame Aktivitäten wie Laufen oder Badminton mit den Arbeitskollegen mit.

Ich stehe Ihnen sofort zur Verfügung und freue mich sehr auf Ihre Einladung zum Vorstellungsgespräch.

\vspace{1em}

Mit freundlichen Grüßen

%\includegraphics[scale=0.2]{./signaturen/NgocMinhDao-Signatur.pdf}\\
Ngoc Minh Dao

Anlagen
\end{document}
