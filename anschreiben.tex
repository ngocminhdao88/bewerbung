\documentclass[11pt,a4paper]{letter}
 
\usepackage[utf8]{inputenc} % for unicode typing
\usepackage[ngerman]{babel} % Neue Deutsch-Rechtschreibung
\usepackage[T1]{fontenc} % fonts encoding
\usepackage[]{newtxtext} % arial font

\usepackage{graphicx} % insert image and picture
\usepackage[left=2.5cm,
            right=2cm,
            top=2.5cm,
            bottom=2cm]{geometry} % define pagelayout

% using microtype package, so to text could be better format
\usepackage[activate={true,nocompatibility},
            final,
            tracking=true,
            kerning=true,
            spacing=true,
            factor=1100,
            stretch=10,
            shrink=10]{microtype}

\usepackage{bookmark} % create a custom bookmark

\setlength{\parskip}{1em} % set the space between two paragraph
\setlength{\parindent}{0pt} % set the indent space of paragraph
\setcounter{secnumdepth}{2} % set the depth of the document, section

\usepackage{xargs} % use argument in new command
\usepackage[table,dvipsnames]{xcolor} % add color support for text

% TODO notes
\usepackage[colorinlistoftodos,prependcaption,textsize=footnotesize]{todonotes} % write TODO
\newcommandx{\unsure}[2][1=]{\todo[linecolor=red, backgroundcolor=red!25, bordercolor=red, #1]{#2}}
\newcommandx{\change}[2][1=]{\todo[linecolor=blue, backgroundcolor=blue!25, bordercolor=blue, #1]{#2}}
\newcommandx{\info}[2][1=]{\todo[linecolor=OliveGreen, backgroundcolor=OliveGreen!25, bordercolor=OliveGreen, #1]{#2}}
\newcommandx{\improvement}[2][1=]{\todo[linecolor=Plum, backgroundcolor=Plum!25, bordercolor=Plum, #1]{#2}}

% macros, define in documents
\newcommand{\JobName}{Entwicklung Doppelkupplung}
\newcommand{\JobBereich}{}
\newcommand{\JobRefNummer}{DE-A-BHL-16-02174}

\newcommand{\FirmaName}{LuK GmbH \& Co. KG}
\newcommand{\FirmaAdresseLineOne}{Industriestraße 3}
\newcommand{\FirmaAdresseLineTwo}{77815 Bühl}

\newcommand{\AnsprechpartnerVoll}{Frau Petra Nagel}
\newcommand{\Ansprechpartner}{Schoeneborn}
\newcommand{\Ansprechpartnerin}{Nagel}

\newcommand{\Beginntermin}{Oktober. 2017}

\begin{document}
% Manual import bookmark
\bookmark[page=1, level=-2]{Anschreiben}
% pagestyle empty means no page number
\pagestyle{empty}

% Sender
\begin{flushleft}
    \textbf{Ngoc Minh Dao}\\
    Obermichelbacher Str. 2\\
    90587 Obermichelbach\\
    E-Mail: ngocminhdao88@gmail.com\\
    Tel.: (+49) 176 74561589
\end{flushleft}

% Empfänger
\begin{flushleft}
    \textbf{\FirmaName}\\
    \AnsprechpartnerVoll\\
    \FirmaAdresseLineOne\\
    \FirmaAdresseLineTwo\\
\end{flushleft}

% Today
\begin{flushright}
    Obermichelbach, \today
\end{flushright}

%\textbf{Bewerbung um eine Abschlussarbeit zum Thema: \JobName~(\JobRefNummer) ab \Beginntermin}
\textbf{Bewerbung um eine Abschlussarbeit im Bereich \JobName}\\
Ref: DE-A-ELG-17-08144
\vspace{1em}

%Sehr geehrte Damen und Herren,
%Sehr geehrter Herr \Ansprechpartner,
Sehr geehrte Frau \Ansprechpartnerin,

schon während meines Studiums des Maschinenbaus konnte ich praktische technische Erfahrung gesammelt.
Als HiWi unterstützte ich das Team des Instituts für Maschinenkonstruktion und Tribologie (IMKT) an der Leibniz Universität Hannover bei der Konstruktionsaufgabe, dem Aufbau sowie der Durchführung von Versuchen für verschiedene Forschungsprojekte.

%\improvement[inline]{Passen die Aufgaben während des Praktikums mit dem Thema?}

Der Schwerpunkt legte ich in meinem Studium auf die Berechnung von Komponenten im Getriebe und die Anwendung von Sensoren in derartigen Systemen.
Neben dem Studium erweiterte ich meine technischen Kenntnisse dadurch, dass ich an Online-Lehrangeboten, Seminaren und Tutorien teilnahm.
Außeruniversitär war ich ein Mitglied in einer akademischen Gruppe für Kraftfahrwesen, wo ich in meiner Freizeit anderen Kommilitonen bei der Wartung bzw. der Reparatur von Fahrzeugen unterstützte.

%\improvement[inline]{Welchen konkreten Schwerpunkt möchte ich mit der Abschlussarbeit bearbeiten?}

Weitere Erfahrung sammelte ich in meinem Praktikum bei Schaeffler Gruppe.
Hier unterstützte ich das Anwendungstechnik Team mit der Berechnung, der Schadensanalyse von Lagern, sowie dem Empfang und dem Büromanagement.
Hierbei konnte ich meine Teamfähigkeit unter Beweis stellen.

Ich spreche gutes Deutsch und habe gute Kenntnisse der Tools Creo, Matlab und MS Office.

Für diese Stelle kann ich sofort antreten.
Gerne überzeuge ich Sie in einem persönlichen Gespräch.

Mit freundlichen Grüßen

\includegraphics[scale=0.2]{./signaturen/NgocMinhDao-Signatur.pdf}\\
Ngoc Minh Dao

Anlagen
\end{document}
